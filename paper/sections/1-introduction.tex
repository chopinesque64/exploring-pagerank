\documentclass[../exploring-pagerank.tex]{subfiles}

\begin{document}
In 1998, only a quarter of commercial Internet search engines could find themselves. Sophisticated -- but fundamentally simplistic -- Boolean search engines could not save impatient searchers from wading through hundreds of results to find a relevant page. Even in 1995, the public Internet contained about 2.5 million pages \cite{TotalNumberWebsites}, but search techniques relied upon classical strategies for collections much smaller and less interconnected than the nascent Internet. Early search engines sorely missed \textit{relevance}. Some, such as AltaVista, used complicated heuristics -- URL length, text style, top-level domain and recency -- as measures of importance. More rigorous vector space models, though they promised refined estimations of a page's relevance to a query, imposed unscalable computation demands.

At this time, however, two Stanford students -- Sergey Brin and Larry Page -- endeavored to bring order to the Web. The students’ graduate project, then hosted at \texttt{google.stanford.edu}, used the link structure of the Web itself to decide how relevant a page was. They viewed links from Web pages as votes on what that page viewed as important on the Web. Combining these votes from all Web pages, they realized, gives a comprehensive picture of the pages that are actually important or essential. Then, a search engine query sorts its content results by this importance, which greatly benefits the likelihood that ``good'' pages are on top.

Brin and Page thus defined their problem as a problem of spectral ranking (eigencentrality). Edmund Landau, in his 1895 paper, formalized the idea of refining a score distribution through an iterative calculation \cite{vignaSpectralRanking2009}. This insight inspired the Landau chess ranking system. Several decades later, when Perron and Frobenius published their theorems, he reformulated the distribution calculation as an eigenproblem. As we will see, Markov chains -- formalized around the same time -- provided a probabalistic interpretation of this eigenproblem. Spectral ranking provides a tractable definition of an entity's relative \textit{importance} within a collection and thus gives an excellent heuristic for its relevance in a search query. In this paper, we explore the foundation of the Google enterprise: PageRank, the algorithm that provides a spectral ranking of all pages on the dynamic Web.
\end{document}
