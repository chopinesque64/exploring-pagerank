\documentclass[11pt,letterpaper]{amsart}
\usepackage[utf8]{inputenc}
\usepackage[english]{babel}
\usepackage{subfiles}
\usepackage{amsmath}
\usepackage{amsfonts}
\usepackage{amssymb}
\usepackage{bbm}
\usepackage{tikz}
\usepackage{mathtools}
\usepackage{subfigure}
\usepackage[hang]{footmisc}

\setlength{\footnotemargin}{2.5mm}
\usetikzlibrary{shapes,automata,arrows}
\graphicspath{{images/}{../images/}}
\setcounter{tocdepth}{1}

\newcommand{\transpose}[1]{#1^{\mathsf{T}}}
\newcommand{\inverse}[1]{#1^{-1}}
\newcommand{\iterate}[2]{#1^{(#2)}}
\newcommand{\parens}[1]{ \left( #1 \right) }
\newcommand{\prob}{\mathbb{P}}
\renewcommand{\thesubtable}{\Roman{subtable}}
\DeclareMathOperator{\rank}{rank}
\DeclareMathOperator{\Null}{null}
\DeclarePairedDelimiterX{\norm}[1]{\lVert}{\rVert}{#1}
\DeclarePairedDelimiter{\abs}{\lvert}{\rvert}

\title{Exploring PageRank}
\author{Nathanael Gentry}
\date{30 April 2019}

\begin{document}
\maketitle
\tableofcontents

\section{Introduction}
\subfile{sections/1-introduction}

\section{First Steps}
\subfile{sections/2-first-steps}

\section{Markov Chains}
\subfile{sections/3-markov-chains}

\section{Perron-Frobenius Theory}
\subfile{sections/4-perron-frobenius}

\section{Limiting Distributions}
\subfile{sections/5-limiting-distributions}

\section{The Google Matrix}
\subfile{sections/6-google-matrix}

\section{Network Topology}
\subfile{sections/7-network-topology}

\section{Further Investigations}
\subfile{sections/8-conclusion}

\newpage
\bibliographystyle{ieeetr}
\bibliography{references}
\end{document}
